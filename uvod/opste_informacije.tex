\ssection{Opšte informacije}
\hspace{2em}\textbf{Internacionalna informatička olimpijada} (IOI, eng. \textit{International Olympiad in \\ Informatics}) najprestižnije je globalno takmičenje učenika osnovnih i srednjih škola iz programiranja. Kao takvo, samo učešće na istoj može biti od velikog značaja za budućnost pojedinca, a osvojene medalje su uspjele plasirati takmičare na prestižne svjetske univerzitete, te im, takođe, otvorile mogućnost zaposlenja u najvećim tehnološkim gigantima. \textbf{IOI} predstavlja cilj određenog dijela takmičara, pretežno iz srednje škole.

Tematika koja se pojavljuje u IOI zadacima definisana je \href{https://ioinformatics.org/page/syllabus/12}{planom}  (eng. \textit{IOI Syllabus}). Dakle, adekvatno bi bilo uključivati one algoritme i strukture podataka, te oblasti, koje su propisane ovim planom.
Iako su zadaci prezentovani na \textbf{IOI}, većinu vremena, ad-hoc prirode (onakve da ne pripadaju ni jednoj kategoriji), oni nisu striktno ad-hoc prirode - kombinuju razne metode kako bi se došlo do rješenja, nerijetko su implementacijski kompleksni i zahtijevaju duboko poznavanje algoritama i struktura podataka.
Zbog ovoga, smatramo da je potrebno da zadaci postavljeni na takmičenjima sadrže više konkretnih algoritama i struktura podataka, koje je vješto potrebno adaptirati i kombinovati sa drugim tehnikama. Ovo ne znači da svaki zadatak mora biti moguće riješiti nekom konkretnom metodom, već da je samo potrebno povećati količinu i nivo poznavanja popularnih algoritama i struktura podataka na višim nivoima takmičenja.