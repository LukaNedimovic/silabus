\ssection{O strukturi dokumenta}
\hspace{2em}Informacije o temama, kao i resursi, nalaze se u 3 kategorije: Matematika, Računarske Nauke - Algoritmi i Računarske Nauke - Strukture Podataka.

Razlog ove podijele baziran je na ideji da bi, u početku, trebalo izučiti, makar na osnovnom nivou, razne teorijske osnove računarskih nauka. Takmičar će brže razumijevati dalje konkretne koncepte vezane za ove teorijske ideje, te biti versatilniji u njihovom korištenju. Što je snažniji matematički alat - kreativnost na takmičenjima će rasti. \\
Nakon toga, algoritmi i strukture podataka se mogu izučavati u generalno arbitrarnom redoslijedu, tj. mogu se kombinovati i učiti po učenikovoj preferenci.

Teme označene sa \textsuperscript{\textnormal{REG}} označavaju teme koje se mogu pojaviti na regionalnom takmičenju, dok teme označene sa \textsuperscript{\textnormal{REP}} označavaju teme koje se mogu pojaviti isključivo na republičkom takmičenju.
Jedna zvjezdica pored nivoa (\reg, \rep) označava teme koje bi se trebale poznavati na generalnom, teorijskom nivou, dok dvije zvjezdice (\regg, \repp) označavaju teme za koje je potrebno poznavanje implementacije.