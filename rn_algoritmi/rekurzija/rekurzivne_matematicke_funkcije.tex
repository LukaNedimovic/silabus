\sssection{\regg Rekurzivne matematičke funkcije (en. \textit{Recursive Mathematical Functions})}

\link{Fibonačijev niz}{Formula za n-ti član Fibonačijevog niza je primjer rekurzivne matematičke funkcije}{https://sr.wikipedia.org/sr-ec/\%D0\%A4\%D0\%B8\%D0\%B1\%D0\%BE\%D0\%BD\%D0\%B0\%D1\%87\%D0\%B8\%D1\%98\%D0\%B5\%D0\%B2_\%D0\%BD\%D0\%B8\%D0\%B7} \\
\link{Faktorijel}{$n! = n \cdot (n-1)!$}{https://www.geeksforgeeks.org/program-for-factorial-of-a-number/} \\
\link{Rekurzivna formula za binomne koeficijente}{
$
\binom{n}{k} = \binom{n-1}{k-1} + \binom{n-1}{k}
$
}{https://www.geeksforgeeks.org/binomial-coefficient-dp-9/} \\